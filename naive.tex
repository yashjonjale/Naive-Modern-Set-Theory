\documentclass[11pt, a4paper]{article}
\usepackage{amsmath, amssymb, amsthm}
\usepackage[margin=1in]{geometry}

% \title{A Report on Foundational Concepts in Set Theory}
\author{Based on the work of P.R. Halmos and others}
\date{}

\begin{document}
% \maketitle

% \section{Introduction}
% Set theory serves as the fundamental bedrock upon which modern mathematics is built. Its concepts provide the language and tools to define and construct nearly all mathematical objects, from numbers and functions to complex geometric spaces. This report provides an overview of the essential pillars of set theory. We begin with the intuitive or "naive" axioms that allow for the construction of basic set-theoretic objects. We then explore the crucial structures of order, which are essential for comparing and organizing elements. This leads to a discussion of the Axiom of Choice and its powerful equivalents, such as Zorn's Lemma, which are indispensable for handling infinite sets.

% Building on this foundation, we delve into the theory of well-ordered sets, which enables the powerful methods of transfinite induction and recursion. These methods are key to understanding the transfinite numbers: the ordinals, which extend the concept of counting to infinite collections, and the cardinals, which measure the size of sets. Finally, we present the formal Zermelo-Fraenkel axioms (ZFC), which provide a rigorous and consistent foundation for the entire theory, resolving the paradoxes that arose from a purely intuitive approach.

\section{The Intuitive Axioms of Set Theory}
The initial development of set theory relies on a set of principles that formalize our intuitive understanding of how sets behave and how they can be constructed. These axioms form the basis of what is often called "naive set theory."

\begin{itemize}
    \item \textbf{Axiom of Extension:} A set is uniquely determined by its elements. If two sets contain the exact same elements, they are the same set.
    \item \textbf{Axiom of Specification:} Given any set $A$ and a logical condition, we can form a new set consisting of all elements of $A$ that satisfy the condition. This is the primary method for creating subsets.
    \item \textbf{Axiom of Pairing:} For any two sets $a$ and $b$, there exists a set $\{a, b\}$ that contains exactly $a$ and $b$.
    \item \textbf{Axiom of Unions:} For any collection of sets, there exists a single set that contains all the elements from all the sets in the collection.
    \item \textbf{Axiom of Powers:} For any set $E$, its power set, $\mathcal{P}(E)$, which is the set of all subsets of $E$, exists.
    \item \textbf{Axiom of Infinity:} There exists at least one infinite set, specifically a set that contains the empty set and is closed under the successor operation ($x \to x \cup \{x\}$). This guarantees the existence of the set of natural numbers, $\omega$.
    \item \textbf{Axiom of Choice:} For any collection of non-empty sets, it is possible to form a new set by choosing one element from each of the original sets.
    \item \textbf{Axiom of Substitution:} If we have a set $A$ and can define a unique object for each element of $A$, then the collection of all these unique objects also forms a set.
\end{itemize}

\section{Structures of Order}
The concept of order is formalized through relations. A relation $\le$ on a set $X$ is a \textbf{partial order} if it is reflexive ($x \le x$), antisymmetric (if $x \le y$ and $y \le x$, then $x=y$), and transitive (if $x \le y$ and $y \le z$, then $x \le z$). A key feature is that not all elements need to be comparable. If all elements are comparable, the order is \textbf{total} and the set is a \textbf{chain}.

Within a partially ordered set, we distinguish between several key concepts:
\begin{itemize}
    \item \textbf{Minimal vs. Least:} A \textbf{minimal} element has no element strictly smaller than it. A \textbf{least} element is smaller than or equal to all other elements. A least element is unique and minimal, but a set can have many minimal elements.
    \item \textbf{Maximal vs. Greatest:} Dually, a \textbf{maximal} element has no element strictly greater than it. A \textbf{greatest} element is greater than or equal to all other elements.
    \item \textbf{Bounds:} A \textbf{lower bound} of a subset is an element less than or equal to all elements in the subset. The greatest of these is the \textbf{infimum}. Dually, an \textbf{upper bound} is greater than or equal to all elements, and the least of these is the \textbf{supremum}.
\end{itemize}

\section{The Axiom of Choice and Zorn's Lemma}
The Axiom of Choice (AC) is a foundational principle for dealing with infinite sets. It states that for any collection of non-empty sets, a new set can be formed by selecting one element from each. While seemingly obvious for finite collections, for infinite ones it is a powerful, non-constructive assertion. It is equivalent to \textbf{Zorn's Lemma}:
\begin{itemize}
    \item \textbf{Zorn's Lemma:} Let $X$ be a non-empty partially ordered set. If every chain in $X$ has an upper bound in $X$, then $X$ must contain at least one maximal element.
\end{itemize}
Zorn's Lemma does not construct the maximal element but guarantees its existence. It is often a more practical tool in proofs than AC itself. Its primary application in set theory is to prove the \textbf{Well-Ordering Theorem}, which states that any set can be equipped with a well-ordering.

\section{Well-Ordered Sets and Transfinite Methods}
A \textbf{well-ordered set} is a totally ordered set where every non-empty subset has a least element. This is a very strong condition; for instance, the natural numbers $\omega$ are well-ordered, but the integers $\mathbb{Z}$ and real numbers $\mathbb{R}$ are not. The importance of well-ordering lies in its connection to generalized induction and recursion.
\begin{itemize}
    \item \textbf{Principle of Transfinite Induction:} This principle allows proofs to proceed over any well-ordered set. It states that if a property holds for an element whenever it holds for all its predecessors, then it must hold for all elements of the set.
    \item \textbf{Transfinite Recursion Theorem:} This allows for the definition of functions on well-ordered sets, where the value of the function at a point can depend on all its previously defined values. This is essential for defining ordinal arithmetic.
\end{itemize}
The \textbf{Comparability Theorem} for well-ordered sets states that for any two such sets, one must be similar (i.e., have the same order structure) to an initial segment of the other. This allows for a linear comparison of all well-orderings.

\section{Ordinal Numbers and Their Arithmetic}
An \textbf{ordinal number} is the canonical representative of a well-ordering. It is defined as a well-ordered set where every element is the set of all its predecessors. This implies that for ordinals $\alpha$ and $\beta$, $\alpha < \beta$ if and only if $\alpha \in \beta$. The ordinals themselves are well-ordered by this relation.
\begin{itemize}
    \item The natural numbers and the set $\omega$ are all ordinals. Ordinals that are not natural numbers are called \textbf{transfinite}.
    \item Ordinals are either \textbf{successor ordinals} (like $3 = 2 \cup \{2\}$ or $\omega+1 = \omega \cup \{\omega\}$) or \textbf{limit ordinals} (like $0$ and $\omega$), which have no immediate predecessor.
    \item \textbf{Ordinal Arithmetic:} Operations are defined to reflect combinations of well-orderings. Addition corresponds to placing orderings end-to-end, and multiplication to lexicographically ordering a Cartesian product. These operations are associative but famously not commutative. For example, $1 + \omega = \omega$, but $\omega + 1$ is a distinct, larger ordinal.
\end{itemize}
There is no set of all ordinals; the assumption of such a set leads to the \textbf{Burali-Forti Paradox}.

\section{Comparing Set Sizes: The Schröder-Bernstein Theorem}
To compare the sizes of arbitrary sets, we use the concept of injection. We say a set $X$ is \textbf{dominated} by a set $Y$, written $X \preceq Y$, if there is a one-to-one function from $X$ into $Y$. The \textbf{Schröder-Bernstein Theorem} provides the crucial property of antisymmetry for this relation:
\begin{itemize}
    \item \textbf{Theorem:} If $X \preceq Y$ and $Y \preceq X$, then $X$ and $Y$ are equivalent (there exists a bijection between them).
\end{itemize}
This theorem allows for a consistent way to compare the sizes, or \textbf{cardinalities}, of sets. Combined with the Well-Ordering Theorem (which implies any two sets are comparable), it establishes that cardinalities are totally ordered.

\section{The Formal Axioms of Zermelo-Fraenkel Set Theory (ZFC)}
To provide a rigorous foundation and avoid the paradoxes of early set theory, the ZFC axioms were developed. They are the standard for modern mathematics.
\begin{enumerate}
    \item \textbf{Axiom of Extensionality:} Sets with the same elements are equal.
    $$ \forall x \forall y [\forall z (z \in x \iff z \in y) \implies x=y] $$
    \item \textbf{Axiom of Regularity (Foundation):} Every non-empty set has a $\in$-minimal element, preventing infinite descending chains of membership like $x_1 \in x_2 \in x_3 \dots$.
    $$ \forall x [\exists a (a \in x) \implies \exists y (y \in x \land \lnot \exists z (z \in y \land z \in x))] $$
    \item \textbf{Axiom Schema of Specification (Separation):} Allows the formation of subsets based on a logical property.
    $$ \forall z \forall w_1 \dots \forall w_n \exists y \forall x [x \in y \iff (x \in z \land \phi(x, w_1, \dots, w_n))] $$
    \item \textbf{Axiom of Pairing:} Guarantees the existence of pairs $\{x, y\}$.
    $$ \forall x \forall y \exists z (x \in z \land y \in z) $$
    \item \textbf{Axiom of Union:} Guarantees the existence of the union of a set of sets.
    $$ \forall F \exists A \forall Y \forall x [(x \in Y \land Y \in F) \implies x \in A] $$
    \item \textbf{Axiom Schema of Replacement:} A powerful axiom stating that the image of a set under a function is also a set.
    $$ \forall A \forall w_1 \dots \forall w_n [(\forall x \in A \exists! y \phi(x,y,w_1 \dots w_n)) \implies \exists B \forall x \in A \exists y \in B \phi(x,y,w_1 \dots w_n)] $$
    \item \textbf{Axiom of Infinity:} Ensures the existence of an infinite set.
    $$ \exists X (\emptyset \in X \land \forall y (y \in X \implies y \cup \{y\} \in X)) $$
    \item \textbf{Axiom of Power Set:} Ensures the existence of the power set.
    $$ \forall x \exists y \forall z [z \subseteq x \implies z \in y] $$
    \item \textbf{Axiom of Choice:} For any collection of non-empty sets, a choice function exists.
    $$ \forall X [(\emptyset \notin X) \implies \exists f: X \to \bigcup X \forall A \in X (f(A) \in A)] $$
\end{enumerate}

\end{document}

